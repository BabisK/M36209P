%%This is a very basic article template.
%%There is just one section and two subsections.
\documentclass{article}
\usepackage{a4wide}
\usepackage{amsmath}
\usepackage{float}
\usepackage{graphicx}
\usepackage{listings}
\lstset{
    inputencoding=utf8,
}

\begin{document}
\title{Ασκήσεις μελέτης της ενότητας «Συντακτική Ανάλυση»}
\author{Charalampos Kaidos}

\section*{Exercise 9}

For this exercise we have implemented the CYK (Cocke–Younger–Kasami) algorithm
which is a parser for context-free grammars in CNF (Chomsky Normal Form). The
CNF is required because it splits the sequencies it examines in half.

It is a dynamic programing algorithm which in its basic form accepts parameters
that describe the grammar and a sentence and replies whether the sentence
derives from the grammar. It can easily be extended to costruct syntax and
dependency trees, which may be more than one for each sentence. Also it can be
extended to handle probabilistic grammars, which is a form of grammar
description that contains weights (probabilities) on the production rules.

We have created two implementations; one for CYK and one for probabilistic CYK.
It dicides on its own whether the grammar contains weights or not and invokes
the appropriate functions. Both implementations respond whether the given
sentens belongs to the grammar and print out the parser trees. The application
reads files from the filesystem where it expects to find grammar in plain text.
We have included two grammars from the slides, one with weights and one without.

The first grammar, without weights (greek characters missing):
\begin{lstlisting}
S -> V NP
V -> 'θέλω'
NP -> Det Nominal
Det -> 'μια'
Adj -> 'πρωινή'
V -> 'επιθυμώ'
Nominal -> Adj Nominal
Nominal -> 'πτήση'
Adj -> 'απογευματινή'
\end{lstlisting}

To execute the program for this grammar (greek characters missing):
\begin{lstlisting}
python cky.py -g flight_grammar.grm -s "θέλω μια πρωινή πτήση"
\end{lstlisting}

And the result (greek characters missing):
\begin{lstlisting}
[0] θέλω [1] μια [2] πρωινή [3] πτήση [4]
[2] Adj [3] Nominal [4] -> [2] Nominal [4]
[1] Det [2] Nominal [4] -> [1] NP [4]
[0] V [1] NP [4] -> [0] S [4]
The sentence is derived from the grammar
\end{lstlisting}

Here we can see that the sentence is accepted and the syntax tree is printed.
The format of the syntax tree is bottom up, left hand are children and right
hand is the parent node.

The second grammar with weights (greek characters missing):
\begin{lstlisting}
S -> V NP [0.5]
NP -> Det Nominal [1]
Nominal -> Nominal PP [0.2]
Nominal -> 'σπίτι' [0.2] | 'τζάκι' [0.3] | 'μεσίτη' [0.3]
PP -> Prep NP [1]
NPPP -> NP PP [1]
S -> V NPPP [0.5]
V -> 'είδαμε' [1]
Det -> 'το' [1]
Prep -> 'με' [1]
\end{lstlisting}

To execute the program for this grammar (greek characters missing):
\begin{lstlisting}
python cky.py -g house_grammar.grm -s "είδαμε το σπίτι με το τζάκι"
\end{lstlisting}

And the result (greek characters missing):
\begin{lstlisting}
[0] είδαμε [1] το [2] σπίτι [3] με [4] το [5] τζάκι [6]
[1] Det:(1.00) [2] Nominal:(0.20) [3] -> [1] NP:(0.20000) [3]
[4] Det:(1.00) [5] Nominal:(0.30) [6] -> [4] NP:(0.30000) [6]
[0] V:(1.00) [1] NP:(0.20) [3] -> [0] S:(0.10000) [3]
[3] Prep:(1.00) [4] NP:(0.30) [6] -> [3] PP:(0.30000) [6]
[2] Nominal:(0.20) [3] PP:(0.30) [6] -> [2] Nominal:(0.01200) [6]
[1] Det:(1.00) [2] Nominal:(0.01) [6] -> [1] NP:(0.01200) [6]
[1] NP:(0.20) [3] PP:(0.30) [6] -> [1] NPPP:(0.06000) [6]
[0] V:(1.00) [1] NPPP:(0.06) [6] -> [0] S:(0.03000) [6]
[0] V:(1.00) [1] NP:(0.01) [6] -> [0] S:(0.00600) [6]
The sentence is derived from the grammar
\end{lstlisting}

Here we see that beside each node in the tree we see the probability of the
node. For this sentence there are 2 syntax trees with different probabilities as
can be observed beside the starting node 'S'.
\end{document}
